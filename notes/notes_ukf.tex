\documentclass{article}
\usepackage{amsmath}
\usepackage{xparse}

\begin{document}

\section{Designing the Kalman Filter}

\subsection{Problem Definition}
\textbf{Goal}: Estimate $x$ and $v$, where $x$ is the position and $v$ is the velocity, given measurements $z = x + \epsilon_R$, $\epsilon_R \sim \mathcal{N}(0, \sigma_R^2)$.

State vector:

\[
\mathbf{x}_k =
\begin{bmatrix}
x_k \\ v_k
\end{bmatrix}
\]

Measurement:
\[
\mathbf{z}_k =
\begin{bmatrix}
z_k
\end{bmatrix}
\]

\subsection{Time Evolution}

\begin{equation}
x_{k+1} = x_k + v_k \cdot \delta t + \frac{1}{2}a \Delta t^2
\end{equation}

\begin{equation}
v_{k+1} = v_k + a \Delta t
\end{equation}

If we did not include acceleration, then the speed would be fixed to a constant, and we would not allow our object to change speed. Let us now write the time evolution in vector form:

\[
\mathbf{x}_k =
\begin{bmatrix}
1 & \Delta t \\
0 & 1
\end{bmatrix}
\mathbf{x}_{k-1}
+
\begin{bmatrix}
\frac{1}{2} \Delta t^2 \\
\Delta t
\end{bmatrix}
a
\]

and now we define $\mathbf{F}$

\[
\mathbf{F} =
\begin{bmatrix}
1 & \Delta t \\
0 & 1
\end{bmatrix}
\]

and $\mathbf{G}$

\[
\mathbf{G} =
\begin{bmatrix}
\frac{1}{2} \Delta t^2 \\
\Delta t
\end{bmatrix}
\]

So that the equations of motion can be written as

\begin{equation}
\mathbf{x}_{k+1} = \mathbf{F} \mathbf{x}_k + \mathbf{G} \mathbf{u}_k
\end{equation}

we assume that $a \sim \mathcal{N}(0, \sigma_a^2)$, i.e., that the acceleration is normally distributed. The speed can then change by some noise which corresponds to the acceleration.

\subsection{Incorporate Measurements}

Our measurement, $z = x + \epsilon_R$, $\epsilon_R \sim \mathcal{N}(0, \sigma_R^2)$, in vector form reads

\[
\mathbf{z}_k =
\begin{bmatrix}
1 & 0
\end{bmatrix}
mathbf{x}_k + \bm{\epsilon}_k
\]

where

\[
\mathbf{H} =
\begin{bmatrix}
1 & 0
\end{bmatrix}
\]

\subsection{Prediction Step}

In the prediction step, the goal is to propagate the state $\mathbf{x}_k \rightarrow \mathbf{x}_{k+1}$.

If $\mathbf{x}_k$ is a random variable defined as follows:

\begin{equation}
\mathbf{x}_k \sim \mathcal{N}(\hat{\mathbf{x}}_k, \mathbf{P}_k)
\end{equation}

Then the Kalman predict equations are

\begin{equation}
\hat{\mathbf{x}}_{k+1} = \mathbf{F} \hat{\mathbf{x}}_{k}
\end{equation}

\begin{equation}
\mathbf{P}_{k+1} = \mathbf{F} \mathbf{P}_k \mathbf{F}^T + \mathbf{G} \sigma_a^2 \mathbf{G}^T
\end{equation}

\subsection{Measurement Step}

In the measurement step, we incorporate knowledge of $z_k$ into our estimate of the state vector, $\mathbf{x}_k$.
The update equations are as follows:

First, we compute the innovation and the innovation covariance

\begin{equation}
\mathbf{y} = \mathbf{z}_k - \mathbf{H}\hat{\mathbf{x}}_k
\end{equation}

\begin{equation}
\mathbf{S}_k = \mathbf{H}\mathbf{P}_k\mathbf{H}^T + \mathbf{R}
\end{equation}

where $\mathbf{R}$ is given by

\[
\mathbf{R} =
\begin{bmatrix}
\sigma_R^2
\end{bmatrix}
\]

We also need to compute the Kalman gain

\begin{equation}
\mathbf{K} = \mathbf{P}\mathbf{H}^T\mathbf{S}_k^{-1}
\end{equation}

And finally, we update. The new mean, incorporating the measurement, is given by

\begin{equation}
\hat{\mathbf{x}}_{|z} = \hat{\mathbf{x}}_k + \mathbf{K} \mathbf{y}
\end{equation}

And the covariance matrix given the measurement is

\begin{equation}
\mathbf{P}_{|z} = (\mathbf{I} - \mathbf{K}\mathbf{H})\mathbf{P}_k
\end{equation}

\end{document}
